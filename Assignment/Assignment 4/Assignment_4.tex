% Options for packages loaded elsewhere
\PassOptionsToPackage{unicode}{hyperref}
\PassOptionsToPackage{hyphens}{url}
%
\documentclass[
]{article}
\usepackage{amsmath,amssymb}
\usepackage{lmodern}
\usepackage{ifxetex,ifluatex}
\ifnum 0\ifxetex 1\fi\ifluatex 1\fi=0 % if pdftex
  \usepackage[T1]{fontenc}
  \usepackage[utf8]{inputenc}
  \usepackage{textcomp} % provide euro and other symbols
\else % if luatex or xetex
  \usepackage{unicode-math}
  \defaultfontfeatures{Scale=MatchLowercase}
  \defaultfontfeatures[\rmfamily]{Ligatures=TeX,Scale=1}
\fi
% Use upquote if available, for straight quotes in verbatim environments
\IfFileExists{upquote.sty}{\usepackage{upquote}}{}
\IfFileExists{microtype.sty}{% use microtype if available
  \usepackage[]{microtype}
  \UseMicrotypeSet[protrusion]{basicmath} % disable protrusion for tt fonts
}{}
\makeatletter
\@ifundefined{KOMAClassName}{% if non-KOMA class
  \IfFileExists{parskip.sty}{%
    \usepackage{parskip}
  }{% else
    \setlength{\parindent}{0pt}
    \setlength{\parskip}{6pt plus 2pt minus 1pt}}
}{% if KOMA class
  \KOMAoptions{parskip=half}}
\makeatother
\usepackage{xcolor}
\IfFileExists{xurl.sty}{\usepackage{xurl}}{} % add URL line breaks if available
\IfFileExists{bookmark.sty}{\usepackage{bookmark}}{\usepackage{hyperref}}
\hypersetup{
  pdftitle={Assignment 4},
  pdfauthor={Amirthavarshini},
  hidelinks,
  pdfcreator={LaTeX via pandoc}}
\urlstyle{same} % disable monospaced font for URLs
\usepackage[margin=1in]{geometry}
\usepackage{color}
\usepackage{fancyvrb}
\newcommand{\VerbBar}{|}
\newcommand{\VERB}{\Verb[commandchars=\\\{\}]}
\DefineVerbatimEnvironment{Highlighting}{Verbatim}{commandchars=\\\{\}}
% Add ',fontsize=\small' for more characters per line
\usepackage{framed}
\definecolor{shadecolor}{RGB}{248,248,248}
\newenvironment{Shaded}{\begin{snugshade}}{\end{snugshade}}
\newcommand{\AlertTok}[1]{\textcolor[rgb]{0.94,0.16,0.16}{#1}}
\newcommand{\AnnotationTok}[1]{\textcolor[rgb]{0.56,0.35,0.01}{\textbf{\textit{#1}}}}
\newcommand{\AttributeTok}[1]{\textcolor[rgb]{0.77,0.63,0.00}{#1}}
\newcommand{\BaseNTok}[1]{\textcolor[rgb]{0.00,0.00,0.81}{#1}}
\newcommand{\BuiltInTok}[1]{#1}
\newcommand{\CharTok}[1]{\textcolor[rgb]{0.31,0.60,0.02}{#1}}
\newcommand{\CommentTok}[1]{\textcolor[rgb]{0.56,0.35,0.01}{\textit{#1}}}
\newcommand{\CommentVarTok}[1]{\textcolor[rgb]{0.56,0.35,0.01}{\textbf{\textit{#1}}}}
\newcommand{\ConstantTok}[1]{\textcolor[rgb]{0.00,0.00,0.00}{#1}}
\newcommand{\ControlFlowTok}[1]{\textcolor[rgb]{0.13,0.29,0.53}{\textbf{#1}}}
\newcommand{\DataTypeTok}[1]{\textcolor[rgb]{0.13,0.29,0.53}{#1}}
\newcommand{\DecValTok}[1]{\textcolor[rgb]{0.00,0.00,0.81}{#1}}
\newcommand{\DocumentationTok}[1]{\textcolor[rgb]{0.56,0.35,0.01}{\textbf{\textit{#1}}}}
\newcommand{\ErrorTok}[1]{\textcolor[rgb]{0.64,0.00,0.00}{\textbf{#1}}}
\newcommand{\ExtensionTok}[1]{#1}
\newcommand{\FloatTok}[1]{\textcolor[rgb]{0.00,0.00,0.81}{#1}}
\newcommand{\FunctionTok}[1]{\textcolor[rgb]{0.00,0.00,0.00}{#1}}
\newcommand{\ImportTok}[1]{#1}
\newcommand{\InformationTok}[1]{\textcolor[rgb]{0.56,0.35,0.01}{\textbf{\textit{#1}}}}
\newcommand{\KeywordTok}[1]{\textcolor[rgb]{0.13,0.29,0.53}{\textbf{#1}}}
\newcommand{\NormalTok}[1]{#1}
\newcommand{\OperatorTok}[1]{\textcolor[rgb]{0.81,0.36,0.00}{\textbf{#1}}}
\newcommand{\OtherTok}[1]{\textcolor[rgb]{0.56,0.35,0.01}{#1}}
\newcommand{\PreprocessorTok}[1]{\textcolor[rgb]{0.56,0.35,0.01}{\textit{#1}}}
\newcommand{\RegionMarkerTok}[1]{#1}
\newcommand{\SpecialCharTok}[1]{\textcolor[rgb]{0.00,0.00,0.00}{#1}}
\newcommand{\SpecialStringTok}[1]{\textcolor[rgb]{0.31,0.60,0.02}{#1}}
\newcommand{\StringTok}[1]{\textcolor[rgb]{0.31,0.60,0.02}{#1}}
\newcommand{\VariableTok}[1]{\textcolor[rgb]{0.00,0.00,0.00}{#1}}
\newcommand{\VerbatimStringTok}[1]{\textcolor[rgb]{0.31,0.60,0.02}{#1}}
\newcommand{\WarningTok}[1]{\textcolor[rgb]{0.56,0.35,0.01}{\textbf{\textit{#1}}}}
\usepackage{longtable,booktabs,array}
\usepackage{calc} % for calculating minipage widths
% Correct order of tables after \paragraph or \subparagraph
\usepackage{etoolbox}
\makeatletter
\patchcmd\longtable{\par}{\if@noskipsec\mbox{}\fi\par}{}{}
\makeatother
% Allow footnotes in longtable head/foot
\IfFileExists{footnotehyper.sty}{\usepackage{footnotehyper}}{\usepackage{footnote}}
\makesavenoteenv{longtable}
\usepackage{graphicx}
\makeatletter
\def\maxwidth{\ifdim\Gin@nat@width>\linewidth\linewidth\else\Gin@nat@width\fi}
\def\maxheight{\ifdim\Gin@nat@height>\textheight\textheight\else\Gin@nat@height\fi}
\makeatother
% Scale images if necessary, so that they will not overflow the page
% margins by default, and it is still possible to overwrite the defaults
% using explicit options in \includegraphics[width, height, ...]{}
\setkeys{Gin}{width=\maxwidth,height=\maxheight,keepaspectratio}
% Set default figure placement to htbp
\makeatletter
\def\fps@figure{htbp}
\makeatother
\setlength{\emergencystretch}{3em} % prevent overfull lines
\providecommand{\tightlist}{%
  \setlength{\itemsep}{0pt}\setlength{\parskip}{0pt}}
\setcounter{secnumdepth}{-\maxdimen} % remove section numbering
\ifluatex
  \usepackage{selnolig}  % disable illegal ligatures
\fi

\title{Assignment 4}
\author{Amirthavarshini}
\date{}

\begin{document}
\maketitle

Loading the required libraries

\begin{Shaded}
\begin{Highlighting}[]
\FunctionTok{library}\NormalTok{(RSQLite)}
\end{Highlighting}
\end{Shaded}

\hypertarget{note}{%
\subsection{Note:}\label{note}}

\hypertarget{please-make-sure-to-place-mediadb.db-and-holla.assignment_4.rmd-file-in-the-same-location}{%
\subsection{Please make sure to place MediaDB.db and
holla.Assignment\_4.rmd file in the same
location}\label{please-make-sure-to-place-mediadb.db-and-holla.assignment_4.rmd-file-in-the-same-location}}

\hypertarget{if-not-please-provide-the-path-for-mediadb.db-file-before-running-it.}{%
\subsection{If not please provide the path for MediaDB.db file before
running
it.}\label{if-not-please-provide-the-path-for-mediadb.db-file-before-running-it.}}

Connecting to Database

\begin{Shaded}
\begin{Highlighting}[]
\NormalTok{dbcon }\OtherTok{\textless{}{-}} \FunctionTok{dbConnect}\NormalTok{(}\FunctionTok{SQLite}\NormalTok{(), }\StringTok{"MediaDB.db"}\NormalTok{)}
\end{Highlighting}
\end{Shaded}

\hypertarget{assumption}{%
\subsection{Assumption:}\label{assumption}}

\hypertarget{since-the-question-1-doesnt-specify-if-we-need-to-consider-all-products-i-am-not-considering-unique-customers.-i.e-if-a-customer-has-multiple-invoices-his-name-would-appear-multiple-times.-in-case-we-need-to-consider-the-customer-only-once-for-multiple-invoices-we-need-to-add-distinct-in-the-select-statement.}{%
\subsection{Since the question 1 doesn't specify if we need to consider
all products, I am not considering unique customers. i.e if a customer
has multiple invoices, his name would appear multiple times. In case we
need to consider the customer only once for multiple invoices, we need
to add DISTINCT in the select
statement.}\label{since-the-question-1-doesnt-specify-if-we-need-to-consider-all-products-i-am-not-considering-unique-customers.-i.e-if-a-customer-has-multiple-invoices-his-name-would-appear-multiple-times.-in-case-we-need-to-consider-the-customer-only-once-for-multiple-invoices-we-need-to-add-distinct-in-the-select-statement.}}

\begin{enumerate}
\def\labelenumi{\arabic{enumi}.}
\tightlist
\item
  (5 pts) What are the last names and emails of all customer who made
  purchased in the store?
\end{enumerate}

\begin{Shaded}
\begin{Highlighting}[]

\KeywordTok{SELECT}\NormalTok{ c.LastName, c.Email }
\KeywordTok{from}\NormalTok{ customers c }\KeywordTok{INNER} \KeywordTok{JOIN}\NormalTok{ invoices i }
\KeywordTok{ON}\NormalTok{ c.Customerid }\OperatorTok{=}\NormalTok{ i.customerid;   }
\end{Highlighting}
\end{Shaded}

\begin{longtable}[]{@{}ll@{}}
\caption{Displaying records 1 - 10}\tabularnewline
\toprule
LastName & Email \\
\midrule
\endfirsthead
\toprule
LastName & Email \\
\midrule
\endhead
Gonçalves &
\href{mailto:luisg@embraer.com.br}{\nolinkurl{luisg@embraer.com.br}} \\
Gonçalves &
\href{mailto:luisg@embraer.com.br}{\nolinkurl{luisg@embraer.com.br}} \\
Gonçalves &
\href{mailto:luisg@embraer.com.br}{\nolinkurl{luisg@embraer.com.br}} \\
Gonçalves &
\href{mailto:luisg@embraer.com.br}{\nolinkurl{luisg@embraer.com.br}} \\
Gonçalves &
\href{mailto:luisg@embraer.com.br}{\nolinkurl{luisg@embraer.com.br}} \\
Gonçalves &
\href{mailto:luisg@embraer.com.br}{\nolinkurl{luisg@embraer.com.br}} \\
Gonçalves &
\href{mailto:luisg@embraer.com.br}{\nolinkurl{luisg@embraer.com.br}} \\
Köhler &
\href{mailto:leonekohler@surfeu.de}{\nolinkurl{leonekohler@surfeu.de}} \\
Köhler &
\href{mailto:leonekohler@surfeu.de}{\nolinkurl{leonekohler@surfeu.de}} \\
Köhler &
\href{mailto:leonekohler@surfeu.de}{\nolinkurl{leonekohler@surfeu.de}} \\
\bottomrule
\end{longtable}

\hypertarget{note-for-the-below-output-please-click-on-the-black-right-arrow-in-the-output-to-see-title.}{%
\subsection{Note: For the below output, please click on the black right
arrow in the output to see
Title.}\label{note-for-the-below-output-please-click-on-the-black-right-arrow-in-the-output-to-see-title.}}

\begin{enumerate}
\def\labelenumi{\arabic{enumi}.}
\setcounter{enumi}{1}
\tightlist
\item
  (5 pts) What are the names of each albums and the artist who created
  it?
\end{enumerate}

\begin{Shaded}
\begin{Highlighting}[]

\KeywordTok{SELECT}\NormalTok{ Name, Title}
\KeywordTok{FROM}\NormalTok{ artists ar }\KeywordTok{INNER} \KeywordTok{JOIN}\NormalTok{ albums ab}
\KeywordTok{on}\NormalTok{ ar.ArtistId }\OperatorTok{=}\NormalTok{ ab.ArtistId;}
\end{Highlighting}
\end{Shaded}

\begin{longtable}[]{@{}ll@{}}
\caption{Displaying records 1 - 10}\tabularnewline
\toprule
Name & Title \\
\midrule
\endfirsthead
\toprule
Name & Title \\
\midrule
\endhead
AC/DC & For Those About To Rock We Salute You \\
Accept & Balls to the Wall \\
Accept & Restless and Wild \\
AC/DC & Let There Be Rock \\
Aerosmith & Big Ones \\
Alanis Morissette & Jagged Little Pill \\
Alice In Chains & Facelift \\
Antônio Carlos Jobim & Warner 25 Anos \\
Apocalyptica & Plays Metallica By Four Cellos \\
Audioslave & Audioslave \\
\bottomrule
\end{longtable}

\hypertarget{note-since-customerid-is-primary-key-i-am-not-considering-distinct-in-count-as-primary-key-cannot-be-duplicate.}{%
\subsection{Note: Since CustomerID is primary key, I am not considering
DISTINCT in count as primary key cannot be
duplicate.}\label{note-since-customerid-is-primary-key-i-am-not-considering-distinct-in-count-as-primary-key-cannot-be-duplicate.}}

\begin{enumerate}
\def\labelenumi{\arabic{enumi}.}
\setcounter{enumi}{2}
\tightlist
\item
  (10 pts) What are the total number of unique customers for each state,
  ordered alphabetically by state who made at least one purchase?
\end{enumerate}

\begin{Shaded}
\begin{Highlighting}[]

\KeywordTok{select}\NormalTok{ state, }\FunctionTok{count}\NormalTok{(customerid)}
\KeywordTok{from}\NormalTok{ customers}
\KeywordTok{where}\NormalTok{ state }\KeywordTok{is} \KeywordTok{not} \KeywordTok{null}
\KeywordTok{group} \KeywordTok{by}\NormalTok{ state}
\KeywordTok{order} \KeywordTok{by}\NormalTok{ State }\KeywordTok{asc}\NormalTok{;}
\end{Highlighting}
\end{Shaded}

\begin{longtable}[]{@{}lr@{}}
\caption{Displaying records 1 - 10}\tabularnewline
\toprule
State & count(customerid) \\
\midrule
\endfirsthead
\toprule
State & count(customerid) \\
\midrule
\endhead
AB & 1 \\
AZ & 1 \\
BC & 1 \\
CA & 3 \\
DF & 1 \\
Dublin & 1 \\
FL & 1 \\
IL & 1 \\
MA & 1 \\
MB & 1 \\
\bottomrule
\end{longtable}

\begin{enumerate}
\def\labelenumi{\arabic{enumi}.}
\setcounter{enumi}{3}
\tightlist
\item
  (10 pts) Which states have more than 10 unique customers?
\end{enumerate}

\begin{Shaded}
\begin{Highlighting}[]

\KeywordTok{SELECT}\NormalTok{ state}
\KeywordTok{FROM}\NormalTok{ customers}
\KeywordTok{WHERE}\NormalTok{ state }\KeywordTok{is} \KeywordTok{NOT} \KeywordTok{NULL}
\KeywordTok{GROUP} \KeywordTok{BY}\NormalTok{ state}
\KeywordTok{HAVING} \FunctionTok{COUNT}\NormalTok{(CustomerId) }\OperatorTok{\textgreater{}} \DecValTok{10}\NormalTok{;}
\end{Highlighting}
\end{Shaded}

\begin{longtable}[]{@{}l@{}}
\caption{0 records}\tabularnewline
\toprule
State \\
\midrule
\endfirsthead
\toprule
State \\
\midrule
\endhead
\bottomrule
\end{longtable}

\begin{enumerate}
\def\labelenumi{\arabic{enumi}.}
\setcounter{enumi}{4}
\tightlist
\item
  (10 pts) What are the names of the artists who made an album
  containing the substring ``symphony'' in the album title?
\end{enumerate}

\begin{Shaded}
\begin{Highlighting}[]

\KeywordTok{SELECT}\NormalTok{ a.name}
\KeywordTok{FROM}\NormalTok{ artists a }\KeywordTok{INNER} \KeywordTok{JOIN}\NormalTok{ albums ab }\KeywordTok{on} 
\NormalTok{a.ArtistId }\OperatorTok{=}\NormalTok{ ab.ArtistId}
\KeywordTok{WHERE}\NormalTok{ ab.Title }\KeywordTok{like} \StringTok{\textquotesingle{}\%symphony\%\textquotesingle{}}\NormalTok{;}
\end{Highlighting}
\end{Shaded}

\begin{longtable}[]{@{}l@{}}
\caption{4 records}\tabularnewline
\toprule
Name \\
\midrule
\endfirsthead
\toprule
Name \\
\midrule
\endhead
Sergei Prokofiev \& Yuri Temirkanov \\
Otto Klemperer \& Philharmonia Orchestra \\
Adrian Leaper \& Doreen de Feis \\
Berliner Philharmoniker \& Herbert Von Karajan \\
\bottomrule
\end{longtable}

\#\#Note: \#\# For Question 6, I have considered ID instead of names to
avoid various mistakes that arise from writing wrong spelling

\begin{enumerate}
\def\labelenumi{\arabic{enumi}.}
\setcounter{enumi}{5}
\tightlist
\item
  (15 pts) What are the names of all artists who performed MPEG (video
  or audio) tracks in either the ``Brazilian Music'' or the ``Grunge''
  playlists?
\end{enumerate}

\begin{Shaded}
\begin{Highlighting}[]

\KeywordTok{select} \KeywordTok{DISTINCT}\NormalTok{(artists.Name)}
\KeywordTok{FROM}\NormalTok{ artists }\KeywordTok{INNER} \KeywordTok{JOIN}\NormalTok{ albums }\KeywordTok{USING}\NormalTok{ (ArtistId)}
\KeywordTok{INNER} \KeywordTok{JOIN}\NormalTok{ tracks }\KeywordTok{USING}\NormalTok{ (Albumid)}
\KeywordTok{INNER} \KeywordTok{JOIN}\NormalTok{ media\_types }\KeywordTok{USING}\NormalTok{ (MediaTypeid)}
\KeywordTok{INNER} \KeywordTok{JOIN}\NormalTok{ playlist\_track }\KeywordTok{USING}\NormalTok{ (TrackId)}
\KeywordTok{INNER} \KeywordTok{JOIN}\NormalTok{ playlists }\KeywordTok{USING}\NormalTok{ (Playlistid)}
\KeywordTok{WHERE}\NormalTok{ media\_types.MediaTypeId }\KeywordTok{IN}\NormalTok{ (}\DecValTok{1}\NormalTok{,}\DecValTok{3}\NormalTok{)}\KeywordTok{AND}\NormalTok{ (playlists.PlaylistId }\OperatorTok{=} \DecValTok{11} \KeywordTok{or}\NormalTok{ playlists.PlaylistId }\OperatorTok{=} \DecValTok{16}\NormalTok{ )}
\end{Highlighting}
\end{Shaded}

\begin{longtable}[]{@{}l@{}}
\caption{Displaying records 1 - 10}\tabularnewline
\toprule
Name \\
\midrule
\endfirsthead
\toprule
Name \\
\midrule
\endhead
Caetano Veloso \\
Chico Buarque \\
Antônio Carlos Jobim \\
Gonzaguinha \\
Cássia Eller \\
Djavan \\
Elis Regina \\
Gilberto Gil \\
Eric Clapton \\
Jorge Ben \\
\bottomrule
\end{longtable}

\begin{enumerate}
\def\labelenumi{\arabic{enumi}.}
\setcounter{enumi}{6}
\tightlist
\item
  (20 pts) How many artists published at least 10 MPEG audio or video
  tracks?
\end{enumerate}

\begin{Shaded}
\begin{Highlighting}[]

\KeywordTok{SELECT} \FunctionTok{COUNT}\NormalTok{(ArtistId)}
\KeywordTok{FROM}\NormalTok{ (}\KeywordTok{select}\NormalTok{ artists.ArtistId }
\KeywordTok{FROM}\NormalTok{ artists }\KeywordTok{INNER} \KeywordTok{JOIN}\NormalTok{ albums }\KeywordTok{USING}\NormalTok{ (ArtistId)}
\KeywordTok{INNER} \KeywordTok{JOIN}\NormalTok{ tracks }\KeywordTok{USING}\NormalTok{ (Albumid)}
\KeywordTok{INNER} \KeywordTok{JOIN}\NormalTok{ media\_types }\KeywordTok{USING}\NormalTok{ (MediaTypeid)}
\KeywordTok{WHERE}\NormalTok{ media\_types.MediaTypeId }\KeywordTok{IN}\NormalTok{ (}\DecValTok{1}\NormalTok{,}\DecValTok{3}\NormalTok{)}
\KeywordTok{GROUP} \KeywordTok{BY}\NormalTok{ artists.ArtistId}
\KeywordTok{HAVING} \FunctionTok{COUNT}\NormalTok{(tracks.TrackId) }\OperatorTok{\textgreater{}=} \DecValTok{10}\NormalTok{)}
\end{Highlighting}
\end{Shaded}

\begin{longtable}[]{@{}r@{}}
\caption{1 records}\tabularnewline
\toprule
COUNT(ArtistId) \\
\midrule
\endfirsthead
\toprule
COUNT(ArtistId) \\
\midrule
\endhead
114 \\
\bottomrule
\end{longtable}

\begin{enumerate}
\def\labelenumi{\arabic{enumi}.}
\setcounter{enumi}{7}
\tightlist
\item
  (25 pts) What is the total length of each playlist in hours? List the
  playlist id and name of only those playlists that are longer than 2
  hours, along with the length in hours rounded to two decimals.
\end{enumerate}

\begin{Shaded}
\begin{Highlighting}[]

\KeywordTok{SELECT}\NormalTok{ playlist\_track.PlaylistId, playlists.Name, }\FunctionTok{ROUND}\NormalTok{(}\FunctionTok{CAST}\NormalTok{(}\FunctionTok{SUM}\NormalTok{(trackS.Milliseconds) }\KeywordTok{as} \DataTypeTok{double}\NormalTok{)}\OperatorTok{/}\DecValTok{3600000}\NormalTok{,}\DecValTok{2}\NormalTok{) }\KeywordTok{as}\NormalTok{ Hours}
\KeywordTok{FROM}\NormalTok{ tracks }\KeywordTok{INNER} \KeywordTok{JOIN}\NormalTok{ playlist\_track }\KeywordTok{USING}\NormalTok{ (TrackId)}
\KeywordTok{INNER} \KeywordTok{JOIN}\NormalTok{ playlists }\KeywordTok{USING}\NormalTok{ (PlaylistId)}
\KeywordTok{GROUP} \KeywordTok{BY}\NormalTok{ playlist\_track.PlaylistId}
\KeywordTok{HAVING}\NormalTok{ Hours }\OperatorTok{\textgreater{}} \DecValTok{2}
\end{Highlighting}
\end{Shaded}

\begin{longtable}[]{@{}rlr@{}}
\caption{Displaying records 1 - 10}\tabularnewline
\toprule
PlaylistId & Name & Hours \\
\midrule
\endfirsthead
\toprule
PlaylistId & Name & Hours \\
\midrule
\endhead
1 & Music & 243.80 \\
3 & TV Shows & 139.19 \\
5 & 90's Music & 110.75 \\
8 & Music & 243.80 \\
10 & TV Shows & 139.19 \\
11 & Brazilian Music & 2.64 \\
12 & Classical & 6.05 \\
14 & Classical 101 - Next Steps & 2.10 \\
15 & Classical 101 - The Basics & 2.07 \\
17 & Heavy Metal Classic & 2.28 \\
\bottomrule
\end{longtable}

\end{document}
